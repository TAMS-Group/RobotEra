\documentclass[t]{beamer}
%\usepackage{multimedia}		% for movies, sounds, animations...
\usepackage[ngerman]{babel}		% new german
\usepackage[utf8]{inputenc}		% input...

\usepackage[cinacsWZ, blockRound, cinacs, conference]{tamsBeamer}
%-----------------------------------------------------------------------------
%-- options		------------------------------------------------------
%			tams	|	- TAMS		publication
%			cinacs		- CINACS	publication
%			engl		- english strings	[german]
%			uniWZ	|	- uni		watermark
%			tamsWZ	|	- tams+uni	watermark
%			cinacsWZ	- cinacs+uni	watermark
%			secToc	|	- toc repetition at each section
%			secTocA		- -"-, all sections: show
%					  replacement for toc in short docs
%			subsecToc	- toc repetition at each subsection
%			secNum  	- (sub)-section numbering
%			fullstep	- always step through items
%			noFoot		- footline	off
%			noPage		- page numbers	off
%			noAuth		- author	off
%			conference	- footline with \foottitle{...}
%			blockBG	|	- block, example etc. background
%			blockRound	- -"-, rounded+shadow


% fonts definitions			--------------------------------------
% ----------------------------------------------------------------------------
% default: cmss, OT1 fontenc		good with UniHH font "The Sans"
%					-> don't change fonts!
%\usepackage{times}			% other fonts
%\usepackage[T1]{fontenc}		%

% document definitions			--------------------------------------
% ----------------------------------------------------------------------------
\title[Konferenzfolien]			% title		-- option: short
  {Konferenzfolien mit \texttt{beamer}}
\subtitle[Beispiele]			% subtitle	-- option: short
  {Beispiele zum Layout}
\foottitle{Alternatives Layout f\"ur Konferenzen}	% stringdefinition

\author[A.~M\"ader]{Andreas M\"ader}
\email{maeder@informatik.uni-hamburg.de}		% stringdefinition

%\author[AutorA, AutorB]		% author	-- option: short
% {A.~Autor\inst{1} \and B.~Autor\inst{2}}%		-- option: \inst{...}
% style option: [tams] predefines institute...
% or define \institute{...}
%					% \inst{...} for different institutions
%\institute[Universities A and B]	% institution	-- option: short
%{ \inst{1}%
%  University of A\\
%  Department of A
%  \and
%  \inst{2}%
%  University of B\\
%  Department of B}

\date[04-nov-2007]			% event/date	-- option: short
  {04.~November 2007}

\subject{TAMS, LaTeX, Folien}		% subject	-- option for pdf


% document starts here			--------------------------------------
% ----------------------------------------------------------------------------
\begin{document}

% titlepage				--------------------------------------
%\frame[plain]{\titlepage}		% suppress head- and footlines
\frame{\titlepage}

% toc					--------------------------------------
\begin{frame}
  \frametitle{\tocName}
  \tableofcontents
  %\tableofcontents[pausesections]	% step through sections
\end{frame}

% intro / abstract etc.			--------------------------------------
\begin{frame}
  \frametitle{Vorwort}
  Diese Folien sollen einige Optionen des alternatives Layouts zeigen.
  Sie wurden erstellt mit \alert{\texttt{cinacs, cinacsWZ, blockRound,
	conference}}\mnewlin
  Die allgemeine Beschreibung zu \texttt{tamsBeamer} steht in \texttt{demo1...}
\end{frame}


% static Contents			--------------------------------------
% ----------------------------------------------------------------------------
\section{Statische Elemente}

\begin{frame}
  \frametitle{\insertsection}
  \framesubtitle{Listen}
  Die Folien werden mit den normalen \LaTeX{}-Befehlen gesetzt.

  \begin{itemize}
  \item \texttt{itemize}
  \item \ldots
  \end{itemize}

  \begin{enumerate}
  \item \texttt{enumerate}
  \item \ldots
  \end{enumerate}

  \begin{description}
  \item[desc1] \texttt{description}
  \item[\ldots] \ldots
  \end{description}
\end{frame}

\begin{frame}[allowframebreaks]
  \frametitle{\insertsection}
  \framesubtitle{block-"ahnliche Umgebungen}
  \small

  \begin{block}{"Uberschrift}
  \texttt{block} Lorem ipsum dolor sit amet.
  \end{block}

  \begin{alertblock}{"Uberschrift}
  \texttt{alertblock} Lorem ipsum dolor sit amet.
  \end{alertblock}

  \begin{exampleblock}{"Uberschrift}
  \texttt{exampleblock} Lorem ipsum dolor sit amet.
  \end{exampleblock}

  \framebreak
  \begin{definition}
  \texttt{definition} Lorem ipsum dolor sit amet.
  \end{definition}

  \begin{example}
  \texttt{example} Lorem ipsum dolor sit amet.
  \end{example}

  \begin{theorem}
  \texttt{theorem} Lorem ipsum dolor sit amet.
  \end{theorem}

  \begin{proof}
  \texttt{proof} Lorem ipsum dolor sit amet.
  \end{proof}
\end{frame}

\begin{frame}
  \frametitle{\insertsection}
  \framesubtitle{weitere Umgebungen}
  \texttt{figure} und \texttt{table} werden wie gewohnt
	benutzt:\footnote{Unterschriften mit \texttt{caption}.}
  \begin{figure}
    \includegraphics[height=8mm]{logoTAMS}
    \caption{TAMS-Logo}
  \end{figure}

  Mehrspaltige \texttt{column}-Umgebungen:
  \begin{columns}[t]
    \begin{column}{4cm}
	Hier stehen\\zwei Zeilen.
    \end{column}
    \begin{column}{4cm}
	Eine Zeile (top-aligned)
    \end{column}
  \end{columns}
\end{frame}

\begin{frame}[containsverbatim]
  \frametitle{\insertsection}
  \framesubtitle{\texttt{verbatim}-Umgebungen}
  Auch andere Umgebungen sind wie gewohnt definiert:
  \texttt{verbatim}, \texttt{semiverbatim}, \texttt{verse},
  \texttt{quotation} etc.
  \bvskip

  \texttt{semiverbatim} ist eine verbatim-Umgebung in der die
  Sonderzeichen \verb+\ { }+ ihre Bedeutung behalten, so dass
  LaTeX-Befehle innerhalb der Umgebung m"oglich sind.

  \begin{block}{Achtung}
    \texttt{verbatim}-Umgebungen und der \texttt{verb}-Befehl sind nur
    m"oglich, wenn der Frame die Option \texttt{containsverbatim} besitzt:\\
    \verb+\begin{frame}[containsverbatim]+
  \end{block}
\end{frame}

% contactframe				--------------------------------------
% ----------------------------------------------------------------------------
\contactframe[\ldots Danke]

\end{document}
